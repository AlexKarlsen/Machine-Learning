Life below water is a critical ecosystem, that experiences a lot of challenging changes. It so important a challenge, that it has found its' way to the UN Sustainable Goals \cite{UN}. The Finnish Environment Institute \cite{meissner} SYKE claims, that the loss of aquatic biodiversity and associated Ecosystem Services is one of the most pressing problems on Earth. SYKE launched the DETECT project \cite{DETECT}, which aims to identify the various species of aquatic macroinvertebrates by imagery and classification using computer vision and machine learning. \\

This paper is a part of the Computer Vision and Machine Learning course at Aarhus University. It documnets my work on solving the in-class Kaggle Challenge. The aim of the Kaggle challenge is improving the classification accuracy of classifiyng the 29 classes of aquaitc macroinvertebrates. Machine Learning and state-of-the-art Deep Learning approaches utilizing Convolutional Neural Networks, are used to examine the different methods and get hand-on experience with a real classification challenge. \\

The rest of the paper is structured as follows; section \ref{sec:dataset} desribes the data set in more depth and how the structure specific structure of the dataset have been used to improve the accuracy. Section \ref{sec:methods} describes the machine learning and deep learning methods used in the project. Section \ref{sec:results} describes and discusses the results obtained from applying the methods. Section \ref{sec:discussion} is a discussion of the overall results and lastly in section \ref{sec:conclusion} the project is concluded upon.